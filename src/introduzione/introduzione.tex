%! Author = evandro
%! Date = 13/02/21

% Preamble
\providecommand{\report}{..}
\documentclass[../main.tex]{subfiles}
\begin{document}
    \chapter{Introduzione}\label{ch:introduzione}
    \section{Il problema da affrontare}\label{sec:Il-problema-da-affrontare}
    Un' applicazione web a tre livelli è composta da un web server (WB), un application server (AS) e un Database (DB): i servizi sono chiamati in sequenza.
    Ogni server è replicato rispettivamente in C\textsubscript{WS}, C\textsubscript{AS} e C\textsubscript{DB} istanze, ognuna delle quali gira su una macchina diversa che condivide una coda comune di capacità finita K\textsubscript{WS}, K\textsubscript{AS} e K\textsubscript{DB}.
    \newline Tre tipi di richieste popolano il sistema:
    \begin{itemize}
        \item \textbf{N\textsubscript{U}} richieste dell' utente (caratterizzate da un \textit{think time} Z\textsubscript{U});
        \item \textbf{N\textsubscript{S}} requisiti software di chiamata di procedura remota;
        \item \textbf{N\textsubscript{B}} programmi di collaborazione.
    \end{itemize}
    Ogni tipo di richiesta richiede una diversa quantità di tempo da ogni risorsa, secondo una distribuzione esponenziale che segue la media mostrata nella tabella sottostante.
    \begin{table}[h]
        \centering
        \begin{tabular}{|l|c|l|c|}
            \hline
            & \textbf{U} & \textbf{S} & \textbf{B} \\ [0.5ex]
            \hline
            \textbf{WS} & 80         & 15         & 5          \\
            \textbf{AS} & 30         & 25         & 120        \\
            \textbf{DB} & 20         & 50         & 40         \\
            \hline
        \end{tabular}
        \caption{Tempo richiesto da ogni risorsa per tipo di richiesta, espresso in \textit{ms}}
        \label{tab:tempo-richiesto-da-ogni-risorsa-per-tipo-di-richiesta}
    \end{table}

    Le richieste che arrivano ad una stazione piena sono bloccate dopo essere state servite.

    \subsection{Set di parametri}\label{subsec:set-di-parametri}
    \begin{table}[h]
        \centering
        \begin{tabular}{|l|c|l|c|l|c|l|c|}
            \hline
            \textbf{ID} & \textbf{K\textsubscript{WS}} & \textbf{K\textsubscript{AS}} & \textbf{K\textsubscript{DB}}
            & \textbf{N\textsubscript{U}}
            & \textbf{N\textsubscript{S}}
            & \textbf{N\textsubscript{B}}
            & \textbf{Z\textsubscript{U}}\\ [0.5ex]
            \hline
            2           & 8                            & 30                           & 60                           & 10                          & 50                          & 100                         & 2 \textit{min}              \\
            \hline
        \end{tabular}
        \caption{Set di parametri del problema}
        \label{tab:set di parametri}
    \end{table}


    \section{Obiettivo}\label{sec:Obiettivo}
    Vogliamo determinare il numero minimo di repliche C\textsubscript{WS}, C\textsubscript{AS} e C\textsubscript{DB} per ogni server tale che le richieste siano servite in media in meno di 2 \textit{sec}.
    
    \section{Struttura del documento}\label{sec:struttura-del-documento}
    \textbf{Chapter 1: Introduzione.}
    Il capitolo ha lo scopo di illustrare il problema da affrontare, i dati a disposizione e l'obiettivo da raggiungere.
    \\
    \textbf{Chapter 2: Il modello.}
    Si descrivono nel dettaglio le scelte implementative e il modello sviluppato.
    \\
    \textbf{Chapter 3: Analisi delle performance.}
    Si analizza il modello dal punto di vista delle performance; si descrive anche l'approccio utilizzato per raggiungere l'obiettivo prefissato.
    \\
    \textbf{Chapter 4: References.}
\end{document}