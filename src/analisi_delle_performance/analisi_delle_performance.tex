%! Author = evandro
%! Date = 13/02/21

% Preamble
\providecommand{\report}{..}
\documentclass[../main.tex]{subfiles}

\begin{document}
    \chapter{Analisi delle performance}\label{ch:analisi-delle-performance}

    La soluzione per via analitica del problema è impraticabile considerata la natura del modello utilizzato.
    I fattori principali che impediscono l’utilizzo di tecniche analitiche:
    sono:
    \begin{itemize}
        \item violazione della \textit{device homogeneity property}: l'utilizzo di code con capacità limitata non
        rispetta questa assunzione.
    \end{itemize}
    Anche se le restanti assunzioni sono rispettate, la violazione della \textit{device homogeneity property}, non
    permette di avere una rete di code separabili.
    Dunque non si possono utilizzare tecniche come la MVA (che richiederebbe anche un solo server per stazione); si è
    deciso di procedere con la simulazione a eventi discreti di JMT e con lo strumento \textit{what-if analysis}.


    \section{Approccio utilizzato}\label{sec:approccio-utilizzato}

    \subsection{Analisi dei dati a disposizione}\label{subsec:analisi-dei-dati-a-disposizione}
    Ispezionando i dati a disposizione [\ref{tab:tempo-richiesto-da-ogni-risorsa-per-tipo-di-richiesta}],
    [\ref{tab:set di parametri}], si nota che la classe più popolosa presente nel sistema sono i programmi di
    collaborazione (B). La stazione \textit{application server} ha un \textit{service time} medio per la suddetta classe
    modellato con un esponenziale di media 120\textit{ms}, maggior valore tra i dati del problema.
    Dunque ci aspettiamo che la stazione \textit{application server} abbia bisogno di un maggior numero di repliche
    rispetto le altre due stazione.
    Un ragionamento simile può essere condotto per la stazione \textit{database} rispetto la stazione
    \textit{web server}.
    Tali considerazioni però non tengono conto della lunghezza delle code di ogni stazione che incide in modo
    determinante sulle performance di un sistema.
    Per quanto riguarda la topologia del modello, la stazione con il \textit{throughput} minore influenza le
    due stazioni successive essendo il modello "circolare".

    \subsection{Modus operandi}\label{subsec:modus-operandi}
    L'analisi del modello è stata condotta partendo da una configurazione base rispetto al numero di repliche per ogni
    stazioni di coda: \textit{web server}, \textit{application server}, \textit{database}. Ognuna ha inizialmente una
    sola replica.
    Lo step successivo è stato quello di individuare la stazione che agiva da collo di bottiglia per il sistema e di
    aumentarne il numero di repliche.
    Per individuare il collo di bottiglia si è preso in considerazione l'utilizzo di ogni stazione, il valore medio più
    elevato decreta a quale stazione aumentare di uno il numero di repliche.
    Nell'aumentare il numero di repliche, si è anche aggiornata la capacità complessiva della relativa stazione
    \footnote{\textit{JSIMgrapf} richiede di definire la capacità di una stazione di coda come somma di \textit{jobs} in
    coda e \textit{jobs} che sono processati.}.
    L'obiettivo è di avere il \textit{response time} del sistema al di sotto dei 2 \textit{sec}, per cui questi passaggi
    sono stati ripetuti fino al raggiungimento di tale obiettivo.


    \section{Soluzioni ottenute}\label{sec:soluzioni-ottenute}
    In tutte le soluzioni ottenute, gli indici analizzati sono arrivati a convergenza. Per ognuno di essi l'intervallo
    di confidenza è pari a 0,99 mentre l'errore relativo è del 3\%.
    \newline
    Applicando quando detto nella sezione [\ref{subsec:modus-operandi}], la prima configurazione a raggiungere
    l'obiettivo è:
    \begin{table}[h]
        \centering
        \begin{tabular}{|l|c|l|c|l|c}
            \hline
            & \textbf{Utilizzo medio} & \textbf{Utilizzo massimo} & \textbf{numero di repliche (C)}  \\ [0.5ex]
            \hline
            \textbf{WS} & 0.6788                  & 0.6949                    & 1                               \\
            \textbf{AS} & 0,9990                  & 1.0117                    & 7                               \\
            \textbf{DB} & 0.8782                  & 0.8992                    & 4                               \\
            \hline
        \end{tabular}
        \caption{Numero di repliche e utilizzo per ogni stazione coda presente nel sistema, soluzione 1.}
        \label{tab:valori-soluzione-1-del-problema}
    \end{table}

    Con questa configurazione [\ref{tab:valori-soluzione-1-del-problema}], si ottiene un \textit{response time} medio del
    sitema pari a 1999,92 \textit{ms}.
    Sebbene riesca ad essere sotto i 2 \textit{sec} richiesti dal progetto, analizzando il valore massimo, esso è pari a
    2017,92 \textit{ms}, non abbastanza da soddisfare le richieste del problema.
    Per avere un picco massimo del \textit{response time} del sistema sotto i 2 \textit{sec}, bisogna aggiungere un
    ulteriore replica alla stazione del \textit{application server}.
    \newline
    La nuova configurazione raggiunta:

    \begin{table}[h]
        \centering
        \begin{tabular}{|l|c|l|c|l|c}
            \hline
            & \textbf{Utilizzo medio} & \textbf{Utilizzo massimo} & \textbf{numero di repliche (C)}  \\ [0.5ex]
            \hline
            \textbf{WS} & 0.7741                  & 0.7959                    & 1                               \\
            \textbf{AS} & 0,9704                  & 1.0067                    & 8                               \\
            \textbf{DB} & 0.9868                  & 1.0041                    & 4                               \\
            \hline
        \end{tabular}
        \caption{Numero di repliche e utilizzo per ogni stazione coda presente nel sistema, soluzione 2.}
        \label{tab:valori-soluzione-2-del-problema}
    \end{table}

    Con questa nuova configurazione [\ref{tab:valori-soluzione-2-del-problema}], si ottiene un \textit{response time}
    medio del sistema pari a 1704,9896 \textit{ms}; mentre il massimo valore è pari a 1747,1874 \textit{ms}.
    Ambedue i valori sono al di sotto del valore richiesto.


    \section{\textit{What-if analysis} delle soluzioni ottenute}\label{sec:what-if-analysis-delle-soluzioni-ottenute}
    Si è utilizzato lo strumento \textit{What-if analysis} di \textit{JSIMgraph} per valutare il comportamento delle
    soluzioni ottenute.
    Si è deciso di studiare il comportamento del modello rispetto alla classe più popolosa presente nel sistema.\newline
    Quindi si è incrementata la popolazione della classe programmi di collaborazione (B) da un valore iniziale di 50 fino ad
    un valore finale di 120.
    Tali valori sono stati scelti prendendo in considerazione la reale popolazione della classe programmi di
    collaborazione (B): 100.
    Il numero di \textit{steps} è stato selezionato pari a 35, in modo da aggiungere due \textit{jobs} per volta.

    \subsection{Utilizzo delle stazioni di coda soluzione 1}\label{subsec:utilizzo-delle-stazioni-di-coda}
    \begin{figure}[H]
        \centering
        \includegraphics[scale = 0.45]{assets/ws_ut_1.png}\\
        \caption[\textit{Utilizzo} della stazione \textit{web server}]{Utilizzo della stazione \textit{web server} al
        variare di N\textsubscript{B}.}
        \label{fig:utilizzo-ws}
    \end{figure}
    Il grafico mostra una riduzione del utilizzo medio del \textit{web server}, ciò è dovuto alla presenza di un collo
    di bottiglia nel sistema. Il \textit{throughput} di tale stazione incide anche sulle altre stazione limitandone, tra
    le altre cose anche l'utilizzo.

    \begin{figure}[H]
        \centering
        \includegraphics[scale = 0.45]{assets/as_ut_1.png}\\
        \caption[\textit{Utilizzo} della stazione \textit{application server}]{Utilizzo della stazione\textit{
            application server} al variare di N\textsubscript{B}.}
        \label{fig:utilizzo-as}
    \end{figure}
    Il grafico mostra un andamento opposto al precedente grafico [\ref{fig:utilizzo-ws}]; è evidente come l'
    \textit{application server} rappresenti il bottleneck di tale sistema e il suo throughput limitato, limiti anche le
    altre risorse del sistema. L'utilizzo medio infatti parte già da valori elevati (0.8) per arrivare a circa 1.

    \begin{figure}[H]
        \centering
        \includegraphics[scale = 0.45]{assets/db_ut_1.png}\\
        \caption[\textit{Utilizzo} della stazione \textit{database}]{Utilizzo della stazione\textit{
            databse} al variare di N\textsubscript{B}.}
        \label{fig:utilizzo-db}
    \end{figure}
    Il grafico ha un andamento simile al primo grafico mostrato [\ref{fig:utilizzo-ws}], per cui
    considerazioni simili a quelle fatte per il \textit{web server} possono essere ripetute anche per il \textit{database}.

    \subsection{\textit{Throughput} soluzione 1}\label{subsec:throughput-soluzione-1}
    Di seguito sono riportati i grafici relativi al \textit{throughput} della soluzione 1.
    \begin{figure}[H]
        \centering
        \includegraphics[scale = 0.6]{assets/ws_th_1.png}\\
        \caption[\textit{Throughput} del \textit{web server}]{\textit{throughput} del \textit{web server}.}
        \label{fig:throughput-time-ws}
    \end{figure}

    \begin{figure}[H]
        \centering
        \includegraphics[scale = 0.6]{assets/as_th_1.png}\\
        \caption[\textit{Throughput} dell' \textit{application server}]{\textit{throughput} dell' \textit{application server}.}
        \label{fig:throughput-time-as}
    \end{figure}

    \begin{figure}[H]
        \centering
        \includegraphics[scale = 0.6]{assets/db_th_1.png}\\
        \caption[\textit{Throughput} del \textit{database}]{\textit{throughput} del \textit{database}.}
        \label{fig:throughput-time-db}
    \end{figure}
    Tutti e tre i grafici mostrano lo stesso andamento e gli stessi valori al variare di N\textsubscript{B}.
    La topologia della rete e la politica di coda \textit{BAS}, non permettono di avere delle perdite nel numero di
    \textit{jobs} nel sistema, per cui la stazione che agisce da collo di bottiglia serve tutti i lavori in attesa
    limitando anche il \textit{throughput} delle altre due stazioni.

    \subsection{\textit{Residence time} soluzione 1}\label{subsec:residence-time-soluzione-1}
    Di seguito sono riportati i grafici relativi al \textit{residence time} della soluzione 1.
    \begin{figure}[H]
        \centering
        \includegraphics[scale = 0.45]{assets/ws_res_1.png}\\
        \caption[\textit{Residence time} del \textit{web server}]{\textit{residence time} del \textit{web server}.}
        \label{fig:residence-time-ws}
    \end{figure}

    \begin{figure}[H]
        \centering
        \includegraphics[scale = 0.45]{assets/as_res_1.png}\\
        \caption[\textit{Residence time} del \textit{application server}]{\textit{residence time} del \textit{application server}.}
        \label{fig:residence-time-as}
    \end{figure}

    \begin{figure}[H]
        \centering
        \includegraphics[scale = 0.45]{assets/db_res_1.png}\\
        \caption[\textit{Residence time} del \textit{database}]{\textit{residence time} del \textit{database}.}
        \label{fig:residence-time-db}
    \end{figure}

    È interessante notare come il \textit{residence time} delle stazioni \textit{web server} e dell'\textit{application server}
    aumenti all'aumentare di N\textsubscript{B}, mentre quello del \textit{database} diminuisce.\newline
    Sebbene per la stazione \textit{application server}, collo di bottiglia di questa configurazione, l'aumento del
    \textit{residence time} può essere ricondotto ad una condizione di saturazione, discorso differente deve essere
    affrontato per il \textit{web server}.
    L'incremento non è dovuto ad una condizione di saturazione, come mostrano i grafici in sezione
    [\ref{subsec:utilizzo-delle-stazioni-di-coda}] bensì può essere ricondotto alla politica di coda utilizzata:
    \textit{BAS}. Nella stazione precedente al collo di bottiglia (\textit{web server}), il \textit{residence time}
    aumenta tanto quanto il \textit{service time} medio rimasto alla stazione successiva (\textit{application server});
    possiamo definire questo aumento come il \textit{blocking time}.
    Per quanto riguarda il \textit{database}, il \textit{residence time} diminuisce essendo a valle del collo di
    bottiglia.

    \subsection{\textit{Response time} del sistema}\label{subsec:response-time-del-sistema}
    \begin{figure}[H]
        \centering
        \includegraphics[scale = 0.6]{assets/res_sys_1.PNG}\\
        \caption[\textit{Response time} medio del sistema]{\textit{Response time} medio del sistema.}
        \label{fig:response-time-system}
    \end{figure}
    Il grafico mostra un andamento crescente al crescere del numero di \textit{jobs} nel sistema.
    In corrispondenza di N\textsubscript{B} pari a 100 (sulle ascisse 1.6 x 10\textsuperscript{2}) \footnote{sull'asse delle ascisse è riportato
    il numero totale dei \textit{jobs} nel sistema: \textbf{N\textsubscript{U}} + \textbf{N\textsubscript{S}} +
    \textbf{N\textsubscript{B}} = 10 + 50 + 100 = 160}, il valore del \textit{Response time} del sistema è di circa 2
    \textit{sec} come atteso dalle precedenti simulazioni [\ref{tab:valori-soluzione-1-del-problema}].

    \subsection{Soluzione 2}\label{subsec:soluzione-2}
    Come per la prima soluzione trovata [\ref{tab:valori-soluzione-1-del-problema}], anche per la seconda è stata condotta
    la stessa \textit{what-if analysis}.

    \begin{figure}[H]
        \centering
        \includegraphics[scale = 0.6]{assets/ws_ut_2.PNG}\\
        \caption[\textit{Utilizzo} della stazione \textit{web server}, soluzione 2]{Utilizzo della stazione
        \textit{web server} al variare di N\textsubscript{B}, soluzione 2.}
        \label{fig:utilizzo-2-ws}
    \end{figure}

    \begin{figure}[H]
        \centering
        \includegraphics[scale = 0.6]{assets/as_ut_2.PNG}\\
        \caption[\textit{Utilizzo} della stazione \textit{application server}, soluzione 2]{Utilizzo della stazione
        \textit{application server} al variare di N\textsubscript{B}, soluzione 2.}
        \label{fig:utilizzo-2-as}
    \end{figure}

    \begin{figure}[H]
        \centering
        \includegraphics[scale = 0.6]{assets/db_ut_2.PNG}\\
        \caption[\textit{Utilizzo} della stazione \textit{database}, soluzione 2]{Utilizzo della stazione
        \textit{database} al variare di N\textsubscript{B}, soluzione 2.}
        \label{fig:utilizzo-2-db}
    \end{figure}

    I grafici seguono andamenti simili a quelli presentati per la soluzione 1 [\ref{subsec:utilizzo-delle-stazioni-di-coda}] per cui simili cosiderazioni si possono
    applicare anche a questa soluzione.
    Confontando i valori notiamo che l'utilization del \textit{web server} e dell' \textit{application server} sono
    aumentati rispetto alla soluzione 1.
    Questo è giustificato dal fatto che il response time, relativo al sistema, della seconda soluzione è minore.


\end{document}